\section{Geofence Settings: Edit High-Frequency Time Span}

% --Description--
%
\begin{cpart}{Description}
Decide which time of the day when the smartphone should check for a perimeter breach more often than otherwise, which obviously is around the time when the user usually clocks in to the office.
\end{cpart}


% --Actors--
%
\begin{cpart}{Actors}
User
\end{cpart}

% --Preconditions--
%
\begin{cpartList}{Preconditions}
\item The smartphone application is in the foreground.
\item The user is logged in to the application.
\end{cpartList}

% --Basic Flow--
%
\begin{cpartList}{Basic Flow}
  \item The user taps on the settings icon.
  \item The smartphone presents the user with the settings view.
  \item The user locates the slider corresponding the hours of the day, with two arrows corresponding the span.
  \item The user slides the arrows to get the preferred time span.
  \item The smartphone saves the setting.
\end{cpartList}

% --Exception Flows--
%
\begin{cpartList}{Exception Flows}
  \begin{innerList}{4}{a}{The user slides the left arrow to the right of the right arrow.}
    \item The smartphone stops the left arrow from sliding over the right arrow, resulting in a time span of 0 minutes.
  \end{innerList}
  \begin{innerList}{4}{b}{The user slides the right arrow to the left of the left arrow.}
    \item The smartphone stops the right arrow from sliding over the left arrow, resulting in a time span of 0 minutes.
  \end{innerList}
\end{cpartList}

% --Postconditions--
%
\begin{cpart}{Postconditions}
The user's preferred time span for the higher frequency is set and saved in the smartphone.
\end{cpart}

\clearpage
