\section{Geofence Use Case}

\begin{cpart}{Description}
Geofence is the virtual perimeter for a real-world geographic area, according to Wikipedia. In this context, the geofence is the virtual perimeter for when the automatic turn-on feature will be triggered.
\end{cpart}

\begin{cpart}{Actors}
User
\end{cpart}

\begin{cpartList}{Preconditions}
\item The user has activated the geofence function in his or her smartphone.
\item The smartphone's GPS hardware is activated.
\item The application is running, either in foreground or background.
\end{cpartList}

\begin{cpartList}{Basic Flow}
\item The user closes in to the virtual perimeter.
\item The smartphone detects the breach of the virtual perimeter.
\item The smartphone sends a request to the system to turn on the coffee machine.
\item The system sends a notification to the smartphone that the coffee machine has been turned on.
\item The smartphone notifies the user that the coffee machine has been turned on.
\end{cpartList}

\begin{cpartList}{Alternative Flows}
\begin{innerList}{4}{a}{The system sends a notification that the coffee machine already is on.}
\item The use case ends without any other action.
\end{innerList}
\end{cpartList}

\begin{cpart}{Postconditions}
The coffee machine is on.
\end{cpart}
