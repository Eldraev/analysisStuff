\section{Revoke Administrator Rights from User}

% --Description--
%
\begin{cpart}{Description}
If a user that has administrator rights leaves the company, or in any other way makes it unnecessary for them to have administrator rights, then these rights should be revoked.
\end{cpart}


% --Actors--
%
\begin{cpart}{Actors}
Two administrators
\end{cpart}

% --Preconditions--
%
\begin{cpart}{Preconditions}
The administrator is logged into the system and is currently on the administrator page.
\end{cpart}

% --Basic Flow--
%
\begin{cpartList}{Basic Flow}
  \item The administrator finds the other administrator in the administrator list.
  \item The administrator removes the other administrator from the list.
  \item The system notifies the administrator by removing the other administrator from the list.
  \item The system writes a log entry for the activity of the user with the current time and date.
\end{cpartList}

% --Exception Flows--
%
\begin{cpartList}{Exception Flows}
  \begin{innerList}{2}{a}{The administrator tries to remove him-/herself.}
    \item The system notifies the administrator that an admin cannot remove him-/herself.
  \end{innerList}
  \begin{innerList}{2}{b}{The (former) administrator no longer has administrator rights.}
    \item The system notifies the (former) administrator that he/she is missing elevated rights.
  \end{innerList}
\end{cpartList}

% --Postconditions--
%
\begin{cpart}{Postconditions}
The user that got the administrator rights revoked is no longer an administrator.
\end{cpart}

\clearpage
